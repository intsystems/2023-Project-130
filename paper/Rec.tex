\documentclass[12pt, a4paper]{extarticle}

\usepackage{arxiv}

\usepackage[T2A]{fontenc}
\usepackage[utf8]{inputenc}
\usepackage[english, russian]{babel}
% \usepackage{cmap}
\usepackage{url}
\usepackage{booktabs}
\usepackage{nicefrac}
\usepackage{microtype}
\usepackage{lipsum}
\usepackage{graphicx}
\usepackage{subfig}
\usepackage[square,sort,comma,numbers]{natbib}
\usepackage{doi}
\usepackage{multicol}
\usepackage{multirow}
\usepackage{tabularx}

\usepackage{tikz}
\usetikzlibrary{matrix}

% Algorithms
\usepackage{algpseudocode}
\usepackage{algorithm}

%% Шрифты
\usepackage{euscript} % Шрифт Евклид
\usepackage{mathrsfs} % Красивый матшрифт
\usepackage{extsizes} % Возможность сделать 14-й шрифт

\usepackage{makecell} % diaghead in a table
\usepackage{amsmath,amsfonts,amssymb,amsthm,mathtools,dsfont}
\usepackage{icomma}

\newcommand{\bz}{\mathbf{z}}
\newcommand{\bx}{\mathbf{x}}
\newcommand{\by}{\mathbf{y}}
\newcommand{\bv}{\mathbf{v}}
\newcommand{\bw}{\mathbf{w}}
\newcommand{\ba}{\mathbf{a}}
\newcommand{\bb}{\mathbf{b}}
\newcommand{\bp}{\mathbf{p}}
\newcommand{\bq}{\mathbf{q}}
\newcommand{\bt}{\mathbf{t}}
\newcommand{\bu}{\mathbf{u}}
\newcommand{\bs}{\mathbf{s}}
\newcommand{\bT}{\mathbf{T}}
\newcommand{\bX}{\mathbf{X}}
\newcommand{\bZ}{\mathbf{Z}}
\newcommand{\bS}{\mathbf{S}}
\newcommand{\bH}{\mathbf{H}}
\newcommand{\bW}{\mathbf{W}}
\newcommand{\bY}{\mathbf{Y}}
\newcommand{\bU}{\mathbf{U}}
\newcommand{\bQ}{\mathbf{Q}}
\newcommand{\bP}{\mathbf{P}}
\newcommand{\bA}{\mathbf{A}}
\newcommand{\bB}{\mathbf{B}}
\newcommand{\bC}{\mathbf{C}}
\newcommand{\bE}{\mathbf{E}}
\newcommand{\bF}{\mathbf{F}}
\newcommand{\bomega}{\boldsymbol{\omega}}
\newcommand{\btheta}{\boldsymbol{\theta}}
\newcommand{\bgamma}{\boldsymbol{\gamma}}
\newcommand{\bdelta}{\boldsymbol{\delta}}
\newcommand{\bPsi}{\boldsymbol{\Psi}}
\newcommand{\bpsi}{\boldsymbol{\psi}}
\newcommand{\bxi}{\boldsymbol{\xi}}
\newcommand{\bchi}{\boldsymbol{\chi}}
\newcommand{\bzeta}{\boldsymbol{\zeta}}
\newcommand{\blambda}{\boldsymbol{\lambda}}
\newcommand{\beps}{\boldsymbol{\varepsilon}}
\newcommand{\bZeta}{\boldsymbol{Z}}
% mathcal
\newcommand{\cX}{\mathcal{X}}
\newcommand{\cY}{\mathcal{Y}}
\newcommand{\cW}{\mathcal{W}}

\newcommand{\dH}{\mathds{H}}
\newcommand{\dR}{\mathds{R}}
% transpose
\newcommand{\T}{^{\mathsf{T}}}

% \renewcommand{\shorttitle}{\textit{arXiv} Шаблон}
\renewcommand{\epsilon}{\ensuremath{\varepsilon}}
\renewcommand{\phi}{\ensuremath{\varphi}}
\renewcommand{\kappa}{\ensuremath{\varkappa}}
\renewcommand{\le}{\ensuremath{\leqslant}}
\renewcommand{\leq}{\ensuremath{\leqslant}}
\renewcommand{\ge}{\ensuremath{\geqslant}}
\renewcommand{\geq}{\ensuremath{\geqslant}}
\renewcommand{\emptyset}{\varnothing}

\usepackage{hyperref}
% \usepackage[usenames,dvipsnames,svgnames,table,rgb]{xcolor}

\hypersetup{
	unicode=true,
	pdftitle={A template for the arxiv style},
	pdfsubject={q-bio.NC, q-bio.QM},
	pdfauthor={David S.~Hippocampus, Elias D.~Striatum},
	pdfkeywords={First keyword, Second keyword, More},
	colorlinks=true,
	linkcolor=black,        % внутренние ссылки
	citecolor=blue,         % на библиографию
	filecolor=magenta,      % на файлы
	urlcolor=blue           % на URL
}

\graphicspath{{../figures/}}

\usepackage{enumitem} % Для модификаций перечневых окружений

\theoremstyle{definition} % "Определение"
\newtheorem{definition}{Опр.}[section]

\usepackage{etoolbox}

\makeatletter
\expandafter\patchcmd\csname\string\algorithmic\endcsname{\itemsep\z@}{\itemsep=1.5mm}{}{}
\makeatother

% убираем номера страниц
\pagenumbering{gobble}

\begin{document} % конец преамбулы, начало документа

\begin{center}
    \textsf{\textbf{
        Рецензия на рукопись\\
        <<Восстановление снимков фМРТ по просматриваемому видеоряду>>\\
        Киселев~Н.\,С.
    }}
\end{center}

\begin{center}
    \textbf{1. Новизна, актуальность и обоснованность работы}
\end{center}

Явно не указаны новизна работы, её отличие от предыдущих исследований, если таковые имеются. Также стоит добавить возможное практическое применение метода. Указана проблема восстановления зависимости между показаниями датчиков фМРТ и восприятия внешнего мира человеком, но не указана польза решения данной проблемы.

\begin{center}
    \textbf{2. Список ошибок, недочетов и замечаний}
\end{center}

\subsubsection*{Аннотация}

Аннотация даёт представление о работе и основных её аспектах. Возможно стоит добавить в неё несколько слов, связанных с полученными результатами.

\subsubsection*{Введение}

Введение содержит основные термины и описание задачи, а также ссылки на связанные работы.

Замечания: Используются ли методы избавления от шумов из связанных работ в данной работе? Нет описания двух терминов: кодировщик, декодировщик.

\subsubsection*{Постановка задачи}

Постановка задачи ясна, даны описания используемых обозначений.

\subsubsection*{Предлагаемый метод восстановления снимков фМРТ}

Используемые формулы понятны, их вывод и обоснованность -- тоже. Необходимо добавить ссылку или описание архитектуры нейронной сети ResNet152.

\subsubsection*{Вычислительный эксперимент}

Есть подробное описание датасета и предобработки данных, обработки результатов, анализа ошибок и выбора гиперпараметров. Нет описания модели, выполняющей основной эксперимент, и обоснования выбора данной модели. Графики аккуратные и читаемые, обоснование к ним подробное и полное. В таблице с описанием датасета нет единиц измерения размерностей изображения и снимка. Для подтверждения гипотезы инвариантности весов модели относительно человека ожидается больше сравнений и усреднение по ним.

\subsubsection*{Заключение}

В заключении отмечено подтверждение основных гипотез.

\begin{center}
    \textbf{3. Комментарии к коду}
\end{center}

Код разбит на секции, что делает его более аккуратным и читаемым. Есть код базового эксперимента.

Нет документации к классам и функциям, а также описания некоторых блоков кода. Большие куски кода оставлены без комментариев. Нет описания предназначения закомментированных частей кода. Тем не менее, предоставление примера работы даёт возможность использовать код без детального его понимания.

Названия переменных хорошие, дают представление о данных, на которые указывают. Структура кода выдержана в соответствии со стандартом.

\begin{center}
    \textbf{4. Общее мнение о работе}
\end{center}

Работа выполнена аккуратно. Статья даёт полное представление о работе. Её можно понять, не обращаясь к автору за пояснениями. Следует добавить в код комментарии.

\vspace{2cm}

\begin{flushleft}
    Рецензент:\\
    Никитина~М.\,А.
\end{flushleft}

\end{document}
